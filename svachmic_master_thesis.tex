% One-page layout: (proof-)reading on display
%%%% \documentclass[11pt,oneside,a4paper]{book}
% Two-page layout: final printing
\documentclass[11pt,twoside,a4paper]{book}   
%=-=-=-=-=-=-=-=-=-=-=-=--=%
% The user of this template may find useful to have an alternative to these 
% officially suggested packages:
\usepackage[czech, english]{babel}
\usepackage[T1]{fontenc} % pouzije EC fonty 
% pripadne pisete-li cesky, pak lze zkusit take:
% \usepackage[OT1]{fontenc} 
\usepackage[utf8]{inputenc}
%=-=-=-=-=-=-=-=-=-=-=-=--=%
% In case of problems with PDF fonts, one may try to uncomment this line:
%\usepackage{lmodern}
%=-=-=-=-=-=-=-=-=-=-=-=--=%
%=-=-=-=-=-=-=-=-=-=-=-=--=%
% Depending on your particular TeX distribution and version of conversion tools 
% (dvips/dvipdf/ps2pdf), some (advanced | desperate) users may prefer to use 
% different settings.
% Please uncomment the following style and use your CSLaTeX (cslatex/pdfcslatex) 
% to process your work. Note however, this file is in UTF-8 and a conversion to 
% your native encoding may be required. Some settings below depend on babel 
% macros and should also be modified. See \selectlanguage \iflanguage.
%\usepackage{czech}  %%%%%\usepackage[T1]{czech} %%%%[IL2] [T1] [OT1]
%=-=-=-=-=-=-=-=-=-=-=-=--=%

%%%%%%%%%%%%%%%%%%%%%%%%%%%%%%%%%%%%%%%
% Styles required in your work follow %
%%%%%%%%%%%%%%%%%%%%%%%%%%%%%%%%%%%%%%%
\usepackage{graphicx}
%\usepackage{indentfirst} %1. odstavec jako v cestine.
\usepackage{setspace}

\usepackage{misc/k336_thesis_macros} % specialni makra pro formatovani DP a BP
 % muzete si vytvorit i sva vlastni v souboru k336_thesis_macros.sty
 % najdete  radu jednoduchych definic, ktere zde ani nejsou pouzity
 % napriklad: 
 % \newcommand{\bfig}{\begin{figure}\begin{center}}
 % \newcommand{\efig}{\end{center}\end{figure}}
 % umoznuje pouzit prikaz \bfig namisto \begin{figure}\begin{center} atd.

\newcommand\TypeOfWork{Master's Thesis}   \typeout{Master's Thesis} 
\newcommand\StudProgram{Open Informatics}
\newcommand\StudBranch{Software Engineering}

%%%%%%%%%%%%%%%%%%%%%%%%%%%%%%%%%%%%%%%%%%%%
% Vyplnte nazev prace, autora a vedouciho
% Set up Work Title, Author and Supervisor
%%%%%%%%%%%%%%%%%%%%%%%%%%%%%%%%%%%%%%%%%%%%

\newcommand\WorkTitle{Semantic Data Analysis and Visualization of User Interactions}
\newcommand\FirstandFamilyName{Bc. Michal Švácha}
\newcommand\Supervisor{Ing. Ivo Malý Ph.D.}


% Pouzijete-li pdflatex, tak je prijemne, kdyz bude mit vase prace
% funkcni odkazy i v pdf formatu
\usepackage[
pdftitle={\WorkTitle},
pdfauthor={\FirstandFamilyName},
bookmarks=true,
colorlinks=true,
breaklinks=true,
urlcolor=red,
citecolor=blue,
linkcolor=blue,
unicode=true,
]
{hyperref}




\begin{document}

%%%%%%%%%%%%%%%%%%%%%%%%%%%%%%%%%%%%%
% Zvolte jednu z moznosti 
% Choose one of the following options
%%%%%%%%%%%%%%%%%%%%%%%%%%%%%%%%%%%%%
%\selectlanguage{czech}
\selectlanguage{english} 

% prikaz \typeout vypise vyse uvedena nastaveni v prikazovem okne
% pro pohodlne ladeni prace


\iflanguage{czech}{
	 \typeout{************************************************}
	 \typeout{Zvoleny jazyk: cestina}
	 \typeout{Typ prace: \TypeOfWork}
	 \typeout{Studijni program: \StudProgram}
	 \typeout{Obor: \StudBranch}
	 \typeout{Jmeno: \FirstandFamilyName}
	 \typeout{Nazev prace: \WorkTitle}
	 \typeout{Vedouci prace: \Supervisor}
	 \typeout{***************************************************}
	 \newcommand\Department{Katedra počítačů}
	 \newcommand\Faculty{Fakulta elektrotechnická}
	 \newcommand\University{České vysoké učení technické v Praze}
	 \newcommand\labelSupervisor{Vedoucí práce}
	 \newcommand\labelStudProgram{Studijní program}
	 \newcommand\labelStudBranch{Obor}
}{
	 \typeout{************************************************}
	 \typeout{Language: english}
	 \typeout{Type of Work: \TypeOfWork}
	 \typeout{Study Program: \StudProgram}
	 \typeout{Study Branch: \StudBranch}
	 \typeout{Author: \FirstandFamilyName}
	 \typeout{Title: \WorkTitle}
	 \typeout{Supervisor: \Supervisor}
	 \typeout{***************************************************}
	 \newcommand\Department{Department of Computer Science and Engineering}
	 \newcommand\Faculty{Faculty of Electrical Engineering}
	 \newcommand\University{Czech Technical University in Prague}
	 \newcommand\labelSupervisor{Supervisor}
	 \newcommand\labelStudProgram{Study Programme} 
	 \newcommand\labelStudBranch{Field of Study}
}




%%%%%%%%%%%%%%%%%%%%%%%%%%    Poznamky ke kompletaci prace
% Nasledujici pasaz uzavrenou v {} ve sve praci samozrejme 
% zakomentujte nebo odstrante. 
% Ve vysledne svazane praci bude nahrazena skutecnym 
% oficialnim zadanim vasi prace.
{
\pagenumbering{roman} \cleardoublepage \thispagestyle{empty}
\chapter*{Na tomto místě bude oficiální zadání vaší práce}
\begin{itemize}
\item Toto zadání je podepsané děkanem a vedoucím katedry,
\item musíte si ho vyzvednout na studiijním oddělení Katedry počítačů na Karlově náměstí,
\item v jedné odevzdané práci bude originál tohoto zadání (originál zůstává po obhajobě na katedře),
\item ve druhé bude na stejném místě neověřená kopie tohoto dokumentu (tato se vám vrátí po obhajobě).
\end{itemize}
\newpage
}

%%%%%%%%%%%%%%%%%%%%%%%%%%    Titulni stranka / Title page 

\coverpagestarts

%%%%%%%%%%%%%%%%%%%%%%%%%%%    Podekovani / Acknowledgements 

\acknowledgements
\noindent
I would like to thank dearly to my cat, dog and goldfish I never had. You were a true inspiration to me. With you, I would have never made this document happen. Also, my PlayStation 3 proved to be an amazing tool to keep me sane while writing this thesis. Thank you SONY.


%%%%%%%%%%%%%%%%%%%%%%%%%%%   Prohlaseni / Declaration 

\declaration{In Prague on May 27, 2016}


%%%%%%%%%%%%%%%%%%%%%%%%%%%%    Abstract 
 
\abstractpage

Current status quo in data collection is to collect everything that is available. This approach has given birth to a trend of the last couple years - "Big Data". Data that is inconveniently large for processing, interpretation and inference. When a company decides to leverage big data, it usually ends up with having too much information and no real business value. This master thesis is trying to take the opportunity of this disorder and connect the two endpoints together while leveraging big data to provide an end-to-end solution.

The focal part of this thesis is the focus on user interactions - collection of data from mobile applications. Before such collection can even happen, interactions must be defined in the first place - the "what", "when" and "why". Starting with agile product management, over to mobile application engineering to interpreting results as the destination - uniting all steps to form a bigger picture.


\vglue40mm

\noindent{\Huge \textbf{Abstrakt}}
\vskip 2.75\baselineskip

\noindent
Momentální status quo ve sběru dat je sbírání a ukládání všeho, co je k dispozici. Tento přístup dal vzniknout trendu posledních let - "Velká data". Tedy data, která jsou  nepohodlně objemná pro zpracování, výklad, a odvozování závěrů. Když se společnost rozhodne, že chce využít velká data, dopadne to většinou tak, že má příliš mnoho informací bez žádné reálné hodnoty. Tato diplomová práce se snaží využít této příležitosti neuspořádanosti pro spojení dvou konců dohromady a vytvořit tak komplexní řešení.

Těžištěm práce je zaměření se na uživatelské interakce - sběr dat z mobilních aplikací. Než nějaký sběr vůbec nastane, je třeba mít interakce definované - tedy "co", "kdy" a "proč". Počínaje agilním vývojem, přes vývoj mobilních aplikací až k interpretaci výsledků jako konečným bodem - spojení všech kroků k vytvoření uceleného náhledu.


%%%%%%%%%%%%%%%%%%%%%%%%%%%%%%%%  Obsah / Table of Contents 

\tableofcontents


%%%%%%%%%%%%%%%%%%%%%%%%%%%%%%%  Seznam obrazku / List of Figures 

\listoffigures


%%%%%%%%%%%%%%%%%%%%%%%%%%%%%%%  Seznam tabulek / List of Tables

\listoftables


%**************************************************************

\mainbodystarts
% horizontalní mezera mezi dvema odstavci
%\parskip=5pt
%11.12.2008 parskip + tolerance
\normalfont
\parskip=0.2\baselineskip plus 0.2\baselineskip minus 0.1\baselineskip

% Odsazeni prvniho radku odstavce resi class book (neaplikuje se na prvni 
% odstavce kapitol, sekci, podsekci atd.) Viz usepackage{indentfirst}.
% Chcete-li selektivne zamezit odsazeni 1. radku nektereho odstavce,
% pouzijte prikaz \noindent.

%**************************************************************

% Pro snadnejsi praci s vetsimi texty je rozumne tyto rozdelit
% do samostatnych souboru nejlepe dle kapitol a tyto potom vkladat
% pomoci prikazu \include{jmeno_souboru.tex} nebo \include{jmeno_souboru}.
% Napr.:
% \include{1_uvod}
% \include{2_teorie}
% atd...

%*****************************************************************************

\chapter{Introduction}

In the lean/agile product development, it is necessary to have a formalized user feedback loops in place, to measure the product performance against various (quantitative) metrics. Such feedback loops are obtaining the information and statistics regarding user engagement and interaction with the developed product. Are the users using it the way the creator imagined it or did they find any other means of utilizing it? What stops the users from doing the task they intended? Do they get everything they need and at the same time does the creator get what was expected? In other words - are the dominant means of usage incentive compatible for the users? 

Not only that the current solutions for gathering such user data for both quantitative and qualitative metrics work out-of-the-box with low-level semantics only (everything is a general activity on a general resource), but also tend to run on somebody else's servers. What if the product developer is in a highly regulated market, such as the pharmaceuticals, and has to own all their users' data? How can the current solutions' space be utilized and tweaked in order to fit in such schema?

The most problematic issue in large corporations is the disconnectivity of the data. It may physically be all there, however, nobody knows what and how should it be connected in order for it to make sense and drive value. Do I have good data or do I just have petabytes of useless log trace? Am I gathering what the application was intended to do or am I only filling the database with irrelevant garbage information. Most importantly, though - Am I gathering the information I need in a consistent fashion that corresponds to the domain of the shareholders?

Let's draw an analogy here.

\subsection*{Example}

\begin{enumerate}
	\item Top management comes with a need to sell more products on mobile devices.
	\item Project management team takes over and breaks it down to user-stories, tasks etc.
	\item Project management team asks the UX team, the mobile applications team and the backend team to perform their tasks in order to bring the product to life.
	\item By the time the product development is spread in three different teams, it is very much likely, that each team creates their own jargon for every activity performed in the product.
		\begin{itemize}
			\item The UX team is driven by the larger picture so they tend to refer to objects in a highly abstract way.
			\item The backend team is driven by the inner processes happening between the database and the REST API, so they speak their precise technical language.
			\item The mobile applications team is somewhere between, but never really aligned with either of the other teams.
		\end{itemize}
	\item When the product is delivered, the project management team has a really hard time translating reported results into the top management's primary goals. Plus - the larger the teams are, the worse the situation actually gets.
\end{enumerate}

This example isn't a problem in project management itself. The company can use the best project management tools on the market and have the smartest people working on their teams and still encounter this problem on a daily basis. This is a larger problem of domain disconnectivity.

\subsection{Personal interviews}

In order to verify the needs I have conducted several interviews with leaders of various departments. These are their reactions to what bothers them about product monitoring during development phase:

\begin{enumerate}
		\item Associate Director, Applied Technology
		\begin{itemize}
				\item[] "The real problem I see is the fact, that all the information I need is on somebody else's server. We can't store any sensitive, let alone confidential information \emph{somewhere} with some random vendor. It's actually illegal in some countries. Unfortunately, sometimes sensitive data is exactly what we need to obtain from the applications to make an informed decision."
		\end{itemize}	

		\item Associate Director, Mobile and Web
		\begin{itemize}
				\item[] "Our needs for tracking KPIs are variable throughout the time and unfortunately the current tools we use are quite inflexible. Because we are in a regulated market, each change that requires new build of an application takes a longer period of time. And time \emph{is} money."
		\end{itemize}				
		
		\item Mobile Application Development Lead
		\begin{itemize}
				\item[] "I have noticed that the one of the biggest obstacles is how should we name what we measure. I have no vocabulary to help me during the development. The only thing we have is a robust Google Analytics toolkit that only allows us to gather low-level actions. We can log that a button was pressed, but what do we name such action? "Button pressed"?"
		\end{itemize}
		
		\item UX Lead
		\begin{itemize}
				\item[] "Our team looks at the high level needs. We are trying to make the user activities in an application as smooth as possible. When we design a low-fidelity prototype, we know what we want to measure. Having the opportunity to add a high-level KPI would help us a lot in order to gather feedback for our prototypes. We are not programmers, we don't know how to add it to the code, but I would love the idea of including what to measure along with the prototype."
		\end{itemize}
\end{enumerate}

\subsection*{Scope of work}

First, I will present the context of agile software development in a regulated market. Then I will take a look at the current tracking solutions being used in mobile development (focus is on iOS development, but most of the tools are multiplatform solutions) - I will examine and analyze their strengths and weaknesses. 

I will further propose a workflow to fit the needs of a larger company with multiple departments, operating in a regulated market. I will utilize existing tools and build on top of them in order to drive value without reinventing the wheel couple times.

Last I will develop a working PoC (Proof of Concept), verify its functions by user testing and further discuss with the stakeholders again whether or not it is the correct path to uniting all departments and connecting the dots in the data.

\chapter{Analysis}

\section{Opportunity Assessment}
\begin{enumerate}
	\item \emph{What problem are we trying to solve?}
	\item[] The problem is that no platform allows by default storage of user data on custom servers. In a regulatory market it is vital to have that option. Also no higher semantic analysis other than defining in code customizations is provided. Enabling alignment of business language and user interaction reporting is a key part of success. Lack of interchangeability tends to end up with a vendor lock-in.
	
	\item \emph{For whom do we solve that problem?}
	\item[] 	For the entire scope of company, top to bottom. Setting a business goal, aligning it with the needs of managers, developers and users.
	
	\item \emph{How will we measure success?}
	\item[] Having gathered data that is securely placed on custom servers and analysed by NLP tools and vizualized by Kibana/D3.js or any other analytics frameworks.
\end{enumerate}

\section{Existing Tools}

\subsection{Google Analytics}

Google Analytics is the by far most popular and widely used framework for monitoring user interactions in applications. Reports everything the developer wishes to. By default it doesn't report anything - the tool has to be activated on application start and then actions need to be wired to the framework. Action can be either wired up after certain custom occurrence has appeared (pressed button, refreshing data) or simply an identifier can be set to an actionable item (button) - then whatever happens with that item, gets reported.

The dashboard website is very detailed and responsive. All data is nicely visualized in graphs and corresponds well with the whole GA ecosystem. Higher order semantic interpretations are missing, though.

The source code is closed source and all of the statistics run ot Google servers. Data is accessible via REST API, but registration is required. Single user account is free, enterprise account is paid for. Enterprise account does NOT include an opt-out from Google servers.

\subsection{Fabric (formerly Crashlytics)}

Crashlytics was also a star framework. Acquired by Twitter, it is now a vital part all purpose platform - Fabric. Fabric is aside from a reporting tool a full featured developer platform used for variety of tasks and obstacles a developer may face - even beta version distribution that has always been a problem for iOS developers. The Crashlytics reporting has been taken a step further and is not only about reporting crashes. It also reports overall statistics, like Google. One nice feature Fabric has is by acquiring AppSee - a way of visualizing user movement in the app through the usage data. A video of steps users take in their app gives the developer a new perspective on how the app is used.

The source code is closed source and all of the statistics run ot Twitter servers. Data is NOT accessible. The whole platform is free to use. Registration is required along with installation of a custom program to install the framework parts in existing projects.

\subsection{App Pulse (formerly Pronq)}

App Pulse is a "new feature by acquisition" - acquired by HP in 2014, it is now part of the portfolio of the new Hewlett Packard Enterprise. It lets users try out their 30-day trial and then charges for everything (no free version).

App Pulse is very different from GA in a way that it reports everything at all times. The usage is fairly low - tens of kilobytes per week, but it is very thorough. Screen time, actions, movements - it is all there. No setup is required for the start. The SDK they supply simply has to be dragged and dropped in Xcode project and then it starts working out-of-the-box. The only issue is the need to have consistent naming of all views, labels, buttons etc. - as it does everything on its own, without hooking up the actionable items to the framework manually, it can be hard to determine which button was which.

The tools are closed source and all of the statistics run on HP servers. No API is provided.

\subsection{Crittercism}

Crittercism doesn't stand out from previously mentioned tools - has their own servers, SDK and works seemlessly. Has some more benchmarking than others, and seems very enterprise oriented - they enable 3rd party API integration into their system to see performance of other APIs used in the application to really find what can be the bottleneck of the app's performance.

The tools are closed source and all of the statistics run on their servers. API is provided.

\subsection{New Relic}

New Relic somewhat differs in a sense that as the only platform there is a mention on their website about "specific needs" - maybe custom server can be provided. Otherwise it is the same strategy - SDK installed in every app and statistics gets reported periodically. Nice alerting system is optionally provided - when crash occurs, web hook to ticketing system can be defined to streamline bug reports.

The SDKs are open source, the analytics runs on their servers (~ possibly 'not only'). No API is provided.

\subsection{Apple}

The last isn't considered framework, but it should be noticed. As companies fight for data from mobile applications - such as Twitter, who gives out their platform free for everybody, naturally the platform owners strive for keeping all that precious data for themselves. Apple announced at WWDC 2015 new iTunes Connect portal redesign and along with it also a new feature - App Analytics. It is fairly thorough in means of usage, downloads, screen time etc, but overall reporting is still very high level and really far from the code. It seems like it is not meant to be a developer tool at all, because there is no in depth code reporting. There are crashes reported, but not very detailed compared to Fabric.

These statistics are provided to every single developer of iOS apps for free on the iTunes Connect website. There is no framework and naturally Apple keeps all of the data for themselves.

\chapter{Design}

\section{System Architecture}

PICTURE

\subsubsection*{Description}
TBD depending on the picture


\subsection*{Components}

\subsection{Issue Tracker}

JIRA is ...

\subsection{Semantic Data Manager}

This is the most crucial component of the whole solution - connecting JIRA and tracking capabilities. The main task is to obtain actionable items from JIRA and create configurations for application engineers to include in the source code. These configurations should be persisted for the sake of reproducibility. Persistence shouldn't be applied to any metadata (detailed measurement description) as it is subject to change in JIRA. Every issue/ticket has a finite unique ID and that should be the only persisted piece of data.

Connecting to JIRA will be handled via its JIRA Agile REST API. It should automatically retrieve a list of all projects and their subsequent issues/tickets that may or may not contain specific metadata regarding fine-grained measurement demands.

SDM will run as a stand-alone microservice and provide UI for easy measurement configuration, but also REST API, should the UI ever be replaced.


\subsection{Tracking Engine}

As a tracking engine, anything that provides API for data storage/retrieval is good enough. Google Analytics is a good feasible option, but for reasons listed in previous chapter (especially privacy), it is more convenient to use own tracking solution.


\subsection{Tracked Device}

This can be any kind of iOS mobile device - iPhone, iPad, Apple Watch or Apple TV. The key part is, that the data that is being sent to the tracking engine is in sync with the data that the semantic data server collected from JIRA.


\subsection{Statistical Front-end}

This component is the most visible one, because it interprets the collected data.


Analýza a návrh implementace (včetně diskuse různých alternativ a volby implementačního prostředí).

\chapter{Implementation}

\section{Deployment}
// Pretty picture

Everything is deployed in the Amazon Web Services (= AWS) cloud environment.

Semantic Data Manager is deployed in AWS Elastic Beanstalk (= EB), which is ... Database runs on managed Amazon RDS (Relational Database Service), which is fast, secure and scalable deployment of database engine. The default database engine is MySQL.

Tracking Engine is also deployed in AWS EB but uses Aurora as database engine. Aurora is ... and has ... configuration, allowing to have really high traffic and maintain its speed and robustness.

Statistical app ...

\section{Issue Tracker}

The issue tracker used in my work environment is JIRA\footnote{No abbreviation here. It is short for GOJIRA, which is Godzilla in Japanese. Rumor has it that it is because the main competing product is Bugzilla.} by Atlassian. It is a standard project management tool providing bug tracking, issue tracking and many other functions. It is conveniently synergistic with other Atlassian tools such as Confluence (for documentation and wiki) and Bitbucket (formerly Stash), which is a server for version control (Mercurial and Git). The advantage is that all these three components are deployed in a Virtual Private Cloud (= VPC) environment - Amazon Web Services. Thus, as a programmer I have fairly easy access to its internal API without being afraid to leak data where it shouldn't.

\subsection{API}

JIRA REST API is quite a mess, which is because Atlassian didn't develop all their products from scratch (Bitbucket was acquired) and it is still visible that the usage isn't seamless. In order to access Teams, Projects and Issues, two API endpoints have to be used:

\begin{enumerate}
	\item JIRA REST API
	\item JIRA REST AGILE
\end{enumerate}

Both of them are APIs (= Application Programming Interface), but I will use their names to distinguish one from another.Both are similar, but also slightly different from each other. 

To illustrate the subtle differences that drive any software engineer mad:

\begin{itemize}
	\item When JIRA REST API is used to obtain the issues, the resulting array uses pagination, because there could be a lot of issues and loading them all at once could take a significant amount of time. In order to determine whether the array I have is final, a parameter "total" is present in the response. This parameter tells how many issues in total there are. In order to load the whole list, it is necessary to keep track how many there are, and how many are left on the stack.
	
	\item When JIRA REST AGILE is used to obtain the teams, the resulting array also uses pagination. In order to determine whether the array I have is final, a parameter called "isLast" is present in the response, having, surprisingly, a boolean value true/false. Obviously, when the value is false, one has to load the next page with the last index that came before.
\end{itemize}

There are plenty of these little surprises that are so easily breakable with any update of the whole system. I honestly do not know, why it isn't the top priority for Atlassian to unite their APIs.

What struck me most though, is the absence of OAuth or Token-based communication. Every query is done via basic auth. While for development it is fine as it allowed me to quickly prototype on top of the API without the need to develop a complex token manager, for production it is quite inconvenient. Even though SSL certificates are all valid and in place, it simply is a terrible architectural choice to not have a proper way to authenticate other applications using the APIs.

\subsubsection{JQL}

JQL stands for JIRA Query Language\cite{jql}. It enables the API user to query the JIRA knowledge graph and extract information.

\begin{figure}[!ht]
	\centering
	\includegraphics[width=0.5\textwidth]{figures/jql}
    \caption{JQL syntax}
\end{figure}

\begin{enumerate}
	\item {\bf Field} - Fields are different types of information in the system. JIRA fields include priority, fixVersion, issue type, etc.
	\item {\bf Operator} - Operators are the heart of the query. They relate the field to the value. Common operators include equals (=), not equals (!=), less than (<), etc.
	\item {\bf Value} – Values are the actual data in the query. They are usually the item for which we are looking.
	\item {\bf Keyword} – Keywords are specific words in the language that have special meaning. In this post we will be focused on AND and OR.
\end{enumerate}

I used JQL in order to get all issues for a certain project:

\begin{lstlisting}
"https://jiraURL/issues/search?jql=project=SAUI"
\end{lstlisting}

Here I used simple query to search all issues where the project is SAUI (= Semantic Analysis of User Interactions). All URL encoders handle the double "= =" and it has never happened to me, that it would encode the parameters badly.

It can obviously be even more powerful, but I was glad it helped me easily get what I needed.

\subsubsection{Methods used}

All communication is handled via HTTP GET and all responses are in JSON format. Cross-site request forgery (= CSRF/XSRF) token system is disabled.

\begin{enumerate}
	\item To get all teams, method {\bf /board} has to be called on JIRA REST AGILE.
	\item To get all projects, method {\bf /projects} has to be called on JIRA REST API.
	\item To get all issues, method {\bf /search} with JQL query has to be called on JIRA REST API.
\end{enumerate}

Interestingly enough, even though, there are two API endpoints, the data is connected, so no further processing was necessary. It is important to note, that the responses are {\bf very} verbose and it is possible to tell in the query to the server not to send some fields back.

\subsection{DSL}

In order to track more than just the beginning and the end of the workflow defined in the scope of the issue, it is necessary to give the project managers a way to define certain observable keywords to pay attention to. Every issue has a field called "description". Usually it contains some human readable set of instructions. Why not piggyback on that and give it just enough structure to make it also computer readable?

I came up with a simple solution - add keyword "WATCH:" on new line and describe what to observe. Parsing is done line by line where the code searches the line for "WATCH:" (case insensitive) and extracts whatever follows until the end of line or occurrence of another "WATCH:". In order not to make it complex, end of line is the end of any description, it does not carry over to the next line.

\subsubsection{Examples}

Here are some examples how the parser for the DSL works. Validation of this technique will be covered in the Testing chapter.

\subsubsection*{Example 1 - Success}

\begin{lstlisting}
As a user, I want to be able to list all projects in the mobile app currently being tracked along with the number of tracked versions in the tracking system.

WATCH: Number projects expanded to the highest detail
WATCH: Filters used to extract information
\end{lstlisting}

This succeeds perfectly, as it parses everything without any hassle.

\subsubsection*{Example 2 - Success}

\begin{lstlisting}
As a user, I want to be able to list all projects in the mobile app currently being tracked along with the number of tracked versions in the tracking system. 
Watch:     Number projects expanded to the highest detail
Watch:     Filters used to extract information
\end{lstlisting}

This succeeds too, because the search is case insensitive and after the search, white spaces are extracted, so the result is the same as in the previous example.

\subsubsection*{Example 3 - Semi-success}

\begin{lstlisting}
As a user, I want to be able to list all projects in the mobile app currently being tracked along with the number of tracked versions in the tracking system.

WATCH: Number projects expanded to the highest detail, Filters used to extract information
\end{lstlisting}

This is a semi-success, almost a failure, but it still yields all the information that the user wanted. It just isn't nicely separated and would need some changes. The error is visible to the programmer and is easy to fix.

\subsubsection*{Example 4 - Failure}

\begin{lstlisting}
As a user, I want to be able to list all projects in the mobile app currently being tracked along with the number of tracked versions in the tracking system. WATCH: Number projects expanded to the highest detail. WATCH: Filters used to extract information
\end{lstlisting}

This yields only one result - "Number projects expanded to the highest detail.". While it seems like it is a good solution, the fact that it seems that way is unfortunately the worst thing about it - because it is hard to discover that there is an error. The programmer sees one observable action item and doesn't see that some got lost during the process, because it was all on one line.

\section{Semantic Data Manager}
The central part of the project should be robust and reliable. For that reason I chose Java as the main technology. For convenience and standardization of the code-base I opted for Spring Boot framework to help me with bootstrapping the heavy work (scheduling, threading, persistence etc.).

\subsection{Spring Boot}

I first tried to use play2 framework for educational purposes, but I encountered too many obstacles deploying play2 application to AWS: 

\begin{itemize}
	\item It does not support WAR packaging.\footnote{There is an unofficial tool that packages the code in a WAR file, but it is not recommended for production environment. Being constrained by highly regulated market, something that already says that it is not production ready is an instant "No thanks".}
	\item It is not possible to run play2 packages (packaged by Activator tool) on Tomcat Server.
	\item It comes with its own Netty Server, which is really clumsy to set up in AWS environment.
\end{itemize}

All three combined resulted in inability to synchronize the play2 application on port 9000 and NGINX running on port 5000. Unfortunately Netty Server does not support compile-time port configuration and NGINX does not support running Activator to set up the port during run-time, so I had to drop the idea of using play2 as I was simply unable to deploy the application. After researching and discussing with my peers and coworkers I looked up Spring MVC and stumbled upon Spring Boot, also recommended by my classmate. I tried few sample apps and found out it supports WAR packaging, runs natively on Tomcat and comes with almost the same perks like play2. I was ready to give it a try.

\begin{figure}[!ht]
	\centering
	\includegraphics[width=0.5\textwidth]{figures/spring}
    \caption{Spring Boot architecture}
\end{figure}

"Spring Boot aims to make it easy to create Spring-powered, production-grade applications and services with minimum fuss. It takes an opinionated view of the Spring platform so that new and existing users can quickly get to the bits they need."\cite{spring-boot-blog}

Primary goals of Spring Boot are\cite{spring-boot-doc}:

\begin{itemize}
	\item Provide a radically faster and widely accessible getting started experience for all Spring development.
	\item Be opinionated out of the box, but get out of the way quickly as requirements start to diverge from the defaults.
	\item Provide a range of non-functional features that are common to large classes of projects (e.g. embedded servers, security, metrics, health checks, externalized configuration).
	\item Absolutely no code generation and no requirement for XML configuration.
\end{itemize}

The advantages seemed to be strong, development environment was convenient (IntelliJ IDEA native support) and after some validation with Amazon support regarding AWS EB deployment, I was confident this would be a good choice.

// H2 DB, Flyweight + DB schema
// REST API + JSON

\subsection{Deployment}

The code is packaged by Maven and deployed as WAR file to AWS EB instance, running Tomcat 8 server. There is no need to configure anything with regards to basic networking - AWS EB is a back-to-back fully managed Platform as a Service (PaaS).

There are multiple things to consider when deploying to AWS EB, even though it seems "super-easy" in most instructional videos and ads:

\begin{enumerate}
	\item Contrary to programmer's logic, one has to first create an "Application" and under that custom "Environments". In other words, Application is the main context, and environments are servers running in the same context, by default having the permission to communicate between each other.
	\item In Environment setup, we can choose either a Web Server Environment or a Worker Environment. Web Server is the hub of any application and is used in both Semantic Data Manager and Tracking Engine. Workers are only used in Tracking Engine and will be explained later.
	\item Chosen configuration was obviously Tomcat and I also opted for automatic load balancing and scaling.
	\item Opting to automatically create an RDS instance along with the environment is a really bad design. Once an environment is terminated, so is the database instance which causes a serious data loss.
	\item In order to access the servers via SSH, it is necessary to define a EC2 Security Group and assign it accordingly. Otherwise, all outside access is prohibited.
	\item It is also crucial to wisely choose the instance type. I opted for m3.medium, because Java applciations by themselves are quite demanding and I didn't want to risk being on the edge when strange errors occur because of insufficient memory capacity.
	\item Permissions and roles are absolutely crucial when it is desired to connect the Environment with other AWS services, such as object storage (S3 = Simple Storage Service) or a messaging queue (SQS = Simple Queue Service).
\end{enumerate}

 The size of the instance is recommended for any application running JVM. The configuration is Intel Xeon E5-2670 v2 (Ivy Bridge), 4GB SSD storage and 3.75GB RAM. Any additional memory is handled via S3.

// JIRA - pagination problems, service account

\subsection{User Interface}

The component Semantic Data Manager provides REST API to be consumed by any kind of client capable of HTTP requests. Because my specialization are mobile applications, I chose to implement a mobile application for iOS as an administrating user interface for this component.

\subsubsection{Application flow}

SCREENSHOTS

\subsubsection{Technologies used}

Moya, tracking engine

\section{Tracking Engine}

Because the tracking engine was developed after the Semantic Data Manager, I opted to use Java again. 

The code is deployed in the same fashion as the Semantic Data Manager.

\section{Tracked Device}
TBD

\section{Time Series Data}

\subsection{Numerous}
Numerous has two components and uses monolithic git repository to keep everything together. It is designed to be a standalone micro service and can be used outside of the scope of this solution.

\subsubsection{Back-end}
For computational convenience in future development (= data modeling, model training etc.), Python was selected to be the language this component will be written in.

The code is packaged in Docker image and is deployed in two EC2 t2.medium instances - one for the API and one for the database. The main difference between T2 and M3 is the processor computational power - T2 has Dual Core Intel Xeon 3.3GHz with Turbo. Also it is a Burstable Performance Instance, meaning that it is provided with a baseline level of CPU performance with the ability to burst above. T2 instances are for workloads that don't use the full CPU often or consistently, but occasionally need to burst (computing statistical models for example in my case).

\subsubsection{Front-end}
Requirements for mobile client were only for iOS devices. Therefore, as no code portability was required, I opted to go native. For native development, Apple's language Swift is used (as of March 2016, in version 2.2). Swift is a general-purpose, multi-paradigm (both object oriented and functional), compiled programming language. It was first released to support iOS and Mac OS X, now supporting also tvOS (Apple TV 4th generation and newer) and watchOS (Apple Watch). Many more cases of use are coming, because Swift compiler has been open-sourced. It is gaining popularity among non-Apple developers mainly due to its safety, robustness and ease of use.

\subsubsection*{Dependencies}

// CocoaPods

\subsubsection*{User Interface}

DASHBOARD screenshot + DETAIL screenshot

Color scheme of the application was provided by a graphical designer. Most of the graphics is set on code level. I am not in favor of bloated projects because of multitude of png files for every possible device. Code generated graphics may introduce some level of complexity, but the space saved on user's device is more important. Even with app-slicing (method of distribution provided by Apple - only the resources needed for your device are downloaded), the amount of space saved is at least 3MB.

// ios-charts

\subsubsection*{Networking}

// Alamofire + VPN

\subsubsection*{Data Model}

// Realm.IO

\chapter{Testing}

\begin{itemize}
 \item Způsob, průběh a výsledky testování.
 \item Srovnání s existujícími řešeními, pokud jsou známy.
\end{itemize} 

\chapter{Conclusion}

The aim of this work was to identify disconnected data endpoints and figure out a way how to connect them and make use of them. I demonstrated the need on misunderstandings between different teams and pointed out where value could be more driven.

In the second part, I analyzed the existing tools on the market and I assessed their usability in the given context. Not being fully satisfied with their functions, I took the initiative and designed a system that would be in-house, deployed in the AWS VPC environment. 

In the third and fourth part, I designed and developed the system into a fully viable product (= MVP). As I went along, I kept validating my ideas with my peers and supervisors.

In the last part, I set myself a goal to test the system from many aspects - idea, usabilty and performance. While some tests went better than the others, it brought me so many valuable insights and pointed out places with room for improvement that I hadn't even considered before. For that I see the tests as an immense success, regardless of the actual results - it shows that the idea is valid, people are interested in it and there is something to continue working on.

\section{Technical debt}
TBD

\section{Future development}
TBD

\section{Closing remarks}
TBD

%*****************************************************************************

% Seznam literatury je v samostatnem souboru reference.bib. Ten
% upravte dle vlastnich potreb, potom zpracujte (a do textu
% zapracujte) pomoci prikazu bibtex a nasledne pdflatex (nebo
% latex). Druhy z nich alespon 2x, aby se poresily odkazy.

\bibliographystyle{abbrv}
%bibliographystyle{plain}
%\bibliographystyle{psc}
{
\def\CS{$\cal C\kern-0.1667em\lower.5ex\hbox{$\cal S$}\kern-0.075em $}
\bibliography{reference}
}

%*****************************************************************************
\appendix

\chapter{Seznam použitých zkratek}

\begin{description}
	\item[AWS] Amazon Web Services
	\item[EC2] Elastic Compute Cloud
	\item[EB] Elastic Beanstalk
	\item[S3] Simple Storage Service
\end{description}
	
\include{chapters/AX_install}
\include{chapters/AX_cd}

%*****************************************************************************

\include{chapters/00_pokyny}

\end{document}
