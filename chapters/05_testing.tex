\chapter{Testing}

The whole system has many components and it is a widely accepted fact that the more complex the system is, the more prone to errors and bugs it is. In this chapter I will describe the benchmarking tests I performed on the most complex part - the tracking engine and the user tests I performed not only on the front-end applications but also the whole workflow of the system as well.

\section{Benchmark Tests}

To verify that the Tracking Engine handles extensive load it is important to simulate a real world peak that could occur during everyday usage. I identified the biggest bottlenecks to be the Gateway (how many requests can it handle easily?) and the Workers (how big tasks can they manage?).

\subsection{Gateway}

TBD

\subsection{Workers}

TBD

\begin{table}[!ht]
\begin{center}
\begin{tabular}{|c|c|c|c|}
\hline
\textbf{Number of entries} & \textbf{Execution Start} & \textbf{Execution End} & \textbf{Duration} \\
\hline
100 & 15:46:35.966 & 15:46:41.595 & $5.629s$ \\
\hline
500 & 16:06:57.851 & 16:07:15.265 & $17.414s$ \\
\hline
1000 & 15:51:23.311 & 15:51:59.060 & $35.749s$ \\
\hline
5000 & N/A & N/A & N/A \\
\hline
10000 & N/A & N/A & N/A \\
\hline
\end{tabular}
\end{center}
\caption{Worker Benchmarks}
\label{tab:ex_db}
\end{table}

TBD

\subsection{Client-side Compression Rate}

TBD

\begin{table}[!ht]
\begin{center}
\begin{tabular}{|l|l|l|}
\hline
\textbf{Number of entries} & \textbf{Before compression} & \textbf{After compression} \\
\hline
100 & 52642 & 693 \\
\hline
500 & 259042 & 1691 \\
\hline
1000 & 517042 & 2940 \\
\hline
5000 & XXXX & XXXX \\
\hline
10000 & XXXX & XXXX \\
\hline
\end{tabular}
\end{center}
\caption{Client-side Compression Rate}
\label{tab:ex_db}
\end{table}

TBD

\section{User Tests}

I performed user tests in three different ways:

\begin{enumerate}
	\item Personal interviews
	\item Scenario test
	\item Panel discussion
\end{enumerate}

Each tested different aspects of the whole solution and each brought me some new insights into which problems I overlooked during the development.

\subsection{Personal Interviews}

\subsection{Scenario Test}

The scenario test was performed on the administrative application, made for creating configuration files for developers to implement.

\subsubsection{Test Setup}

The test was performed on an iPhone 6S device, running the real application, connected to real data. Issues in JIRA were created by me ready to be used during the test (there was no need for the testers to create those issues, because that is JIRA workflow, not the workflow of the application).

The test was carried out in a semi-informal environment - in the company library on a comfortable sofa, surrounded by bookshelves. I sat in front of the tester, not peaking over his/her shoulder. 

I asked each tester to narrate every step he/she made and point out any inconvenience, should one ever occur. I ensured everybody, that criticism is welcomed and that there is absolutely no problem if they fail performing the task. That is what the test should be about and it is vital that the tester is aware of it.

\subsubsection{Scenario}

The scenario was made up this way:

\begin{enumerate}
	\item Start creating a new configuration.
	\item The configuration should be made for project "Semantic Analysis of User Interactions".
	\item Select to measure the performance of the measurement switch.
	\item Are there any special values to be tested?
	\item Upload the configuration file to Dropbox.
	\item Can you tell me precisely what was in the configuration file again?
\end{enumerate}

The correct path to achieve this was:

\begin{enumerate}
	\item Click on the plus button in the top right-hand corner in the navigation bar.
	\item Scroll to letter "S" and select the project "Semantic Analysis of User Interactions".
	\item Click on the issue "Performance measurement switch".
	\item The correct answer is yes - it is written in the comment below the name of the issue. There are two watched values.
	\item Click on the button "Upload to Dropbox". After clicking done after a successful upload, the application is automatically brought back to the first screen.
	\item Click on the button in the bottom right-hand corner "View Generated Configurations". Select the one with the most recent time by clicking on it and it leads directly to a list with all the issues in the configuration.
\end{enumerate}

How did the testers perform:

\begin{table}[!ht]
\begin{center}
\begin{tabular}{|c|c|c|c|c|c|}
\hline
\textbf{Step} & \textbf{Person 1} & \textbf{Person 2} & \textbf{Person 3} & \textbf{Person 4} & \textbf{Person 5} \\
\hline
\textbf{1.} & 52642 & 693 & XXXXX & XXXXX & XXXXX \\
\hline
\textbf{2.} & 52642 & 693 & XXXXX & XXXXX & XXXXX \\
\hline
\textbf{3.} & 52642 & 693 & XXXXX & XXXXX & XXXXX \\
\hline
\textbf{4.} & 52642 & 693 & XXXXX & XXXXX & XXXXX \\
\hline
\textbf{5.} & 52642 & 693 & XXXXX & XXXXX & XXXXX \\
\hline
\textbf{6.} & 52642 & 693 & XXXXX & XXXXX & XXXXX \\
\hline
\end{tabular}
\end{center}
\caption{Client-side Compression Rate}
\label{tab:ex_db}
\end{table}

\subsection{Panel Discussion}

This test I expected to be the most cruel one - I invited programmers with varying levels of experience and wide range of specializations:

\begin{enumerate}
	\item Person 1 - Technical Lead
	\item Person 2 - C\# enthusiast, Machine Learning Engineer
	\item Person 3 - Polyglot Programmer with experience in every aspect of development
	\item Person 4 - C/C++ Programmer with 20+ years of experience
\end{enumerate}