\chapter{Introduction}

In the lean/agile product development, it is necessary to have a formalized user feedback loops in place, to measure the product performance against various (quantitative) metrics. Such feedback loops are getting information and statistics regarding user engagement and interaction with the product. Are the users using it the way the creator imagined it or did they find any other means of utilizing it? What stops the users from doing the task they intended? Do they get everything they need and at the same time does the creator get what was expected? In other words - are the dominant means of usage incentive compatible for the users? 

Not only that the current solutions for gathering such user data for both quantitative and qualitative metrics work out-of-the-box with low-level semantics only (everything is a general activity on a general resource), but also tend to run on somebody else's servers. What if the product developer is in a highly regulated market, such as pharmaceuticals, and has to own all their users' data? How can the current solutions' space be utilized and tweaked in order to fit in such schema?

The most problematic issue in large corporations is the disconnectivity of the data. It may physically be all there, however, nobody knows what and how should it be connected in order for it to make sense and drive value. Do I have good data or do I just have petabytes of useless log trace? Am I gathering what the application was inteded to do or am I only filling the database of irrelevant information. Most importantly, though - Am I gathering the information I need in a consistent fashion that coresponds to the domain of the shareholders?

Let's draw an analogy here.

\subsection*{Example}

The management comes with a need to create an e-commerce mobile application to drive sales of their products. The main KPI (= Key Performance Indicator) is the customer turnover (= how many incoming customers buy a product) and retention (= how many customers that previously bought a product come back). This idea is passed/assigned to middle management for processing. At this point one can assume that the domain vocabulary for the project is almost the same, as the direct source (management) was the one to mediate the idea. From then on, project is passed on to various people in various departments (design, back-end, front-end). Every department has a different technique of work and may also have different jargon. That has the effect that the first idea may be fulfilled but not properly measured. The management was clearly interested in \emph{customer turnover} and \emph{customer retention}. Those two activities were translated into "buying customer" and "repeated shopping" or any other equivalents. While it does bring the message in the end, it does slow down the whole process that can be automated. Starting from the development and ending with the interpretation for the management.

\subsection*{Scope of work}

First, I will present the context of software development in a regulated market. Then I will take a look at the current tracking solutions being used in mobile development (focus is on iOS development, but most of the tools are multiplatform). I will examine and analyze their strengths and weaknesses and further propose an architecture to fit the needs of a company operating in a regulated market. And last I will propose a working PoC (Proof of Concept) tested in a real environment.