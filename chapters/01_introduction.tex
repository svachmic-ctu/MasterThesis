\chapter{Introduction}

In the lean/agile product development, it is necessary to have a formalized user feedback loops in place, to measure the product performance against various (quantitative) metrics. Such feedback loops are getting information and statistics regarding user engagement and interaction with the product. Is the user using it the way the creator imagined it or did he find any other means of utilizing it? Not only that the current solutions for gathering such user data for both quantitative and qualitative metrics work out-of-the-box with low-level semantics only (everything is a general activity on a general resource), but also tend to run on somebody else's servers. What if the product developer is in a highly regulated market, such as pharmaceuticals, and has to own all their users' data? How can the current solutions' space be utilized and tweaked in order to fit in such schema?

First, let's look at the current solutions being used in mobile development (specifically, iOS jargon is used, but the tools are multiplatform), then I'll examine their strengths and weaknesses. Further I'll try to propose an architecture to fit the needs of interchangeability and as the last point I will propose a prototype interface for interpreting gathered data.



Úvod charakterizující kontext zadání, případně motivace.

Výsledná struktura vaší práce a názvy a rozsahy jednotlivých kapitol se samozřejmě budou lišit podle typu práce a podle konkrétní povahy zpracovávaného tématu. Níže uvedená struktura práce odpovídá \textit{práci implementační}, viz \cite{infodp} respektive \cite{infobp}.