\chapter{Design}

\section{System Architecture}

PICTURE

\subsubsection*{Description}
TBD depending on the picture



\section{Issue Tracker}

...

\subsection{Knowledge structure}

Every company uses a little different way of structuring their knowledge in their project management tools. In my situation, the structure was as follows:

\subsubsection{Teams}

Everybody is distributed into teams in various departments. While departments are important in the company structure, in the project management, teams are more relevant. Every team has its own Project Board (either Scrum or Kanban) that reflects the state of the team in time - how much work has been done, what will be done and what's currently being done.

\subsubsection{Projects}

Each team works on their own projects. Often times, collaboration does occur, but the owner of the project is still the team. If other person collaborates with other team, it is reflected in his/her work statistics  (storypoints, hours spent), but the ownership stays within the scope of the project and thus the team as well.

\subsubsection{Issues}

Issue is the most granular entity in the system. It defines a step or set of steps to be performed and is usually assigned to one person. Workflows for reviewing etc. vary from team to team. There is a possibility to break down the work on single issue (if it's especially large one) to sub-tasks, but those are not given IDs, so they are not unique per se. They only serve for clarity and visibility of work being done during the day. When all sub-tasks are finished, the whole issue is finished. That is the state reflected by the API.

\subsection{Semantic Data Manager}

This is the most crucial component of the whole solution - connecting JIRA and tracking capabilities. The main task is to obtain actionable items from JIRA and create configurations for application engineers to include in the source code. These configurations should be persisted for the sake of reproducibility. Persistence shouldn't be applied to any metadata (detailed measurement description) as it is subject to change in JIRA. Every issue/ticket has a finite unique ID and that should be the only persisted piece of data.

Connecting to JIRA will be handled via its JIRA Agile REST API. It should automatically retrieve a list of all projects and their subsequent issues/tickets that may or may not contain specific metadata regarding fine-grained measurement demands.

SDM will run as a stand-alone microservice and provide UI for easy measurement configuration, but also REST API, should the UI ever be replaced.


\subsection{Tracking Engine}

As a tracking engine, anything that provides API for data storage/retrieval is good enough. Google Analytics is a good feasible option, but for reasons listed in previous chapter (especially privacy), it is more convenient to use own tracking solution.


\subsection{Tracked Device}

This can be any kind of iOS mobile device - iPhone, iPad, Apple Watch or Apple TV. The key part is, that the data that is being sent to the tracking engine is in sync with the data that the semantic data server collected from JIRA.


\subsection{Statistical Front-end}

This component is the most visible one, because it interprets the collected data.